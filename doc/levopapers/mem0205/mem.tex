% memory latencies
%
%\documentclass[10pt,twocolumn,dvips]{article}
\documentclass[10pt,dvips]{article}
\usepackage[english]{babel}
\usepackage{epsfig}
%
%\usepackage{fancyheadings}
%\usepackage[T1]{fontenc}
%\usepackage[latin1]{inputenc}
%
%\usepackage{twocolumn}
%\usepackage{verbatim,moreverb,doublespace}
%\usepackage{rotate,lscape,dcolumn,array,rotating,latexsym}
%
%\input{epsf}
%
\textwidth 6.5in
\textheight 9.0in
\topmargin -0.6in
\oddsidemargin 0mm
\evensidemargin 0mm
%
%\pagestyle{empty}
%
\begin{document}
\parskip 2mm
%
%
\title{Memory Latency Notes}
%
\author{
D. Morano\\
Northeastern University\\
dmorano@ece.neu.edu\\
}
%
\maketitle
%\thispagestyle{empty}
%
% abstract goes here if you have one
%
%
\section{Introduction}
%
These notes include some information on 
Dynamic Random Access Memory (DRAM) latencies
in current, and possibly future, machines.
Some information and background on the increasing 
problem that DRAM latency performance presents to the whole
computer design is also included.
%
\section{Current DRAM Latencies}
%
Cuppu et al \cite{cuppu01high} presents a good analysis of
various DRAM design architectures using the various different DRAM chip
technologies that are currently available or are soon to be
available.  Their analysis largely focuses on the higher-end
workstation design point.
This work was an extension of prior work by Cuppu et ~\cite{cuppu99performance}.
They show
how the chip DRAM timing (precharge, row, column, and data
transfer) for Enhanced Synchronous DRAMs (ESDRAM) 
(among other DRAM chip variants) translates
into average access latencies for a realistic memory system design.  
They show
that 100 MHz ESDRAM (a very conservative design choice at the present time) 
translates into average main
memory latencies of between 120 and 170 nsecs for the benchmarks that
they investigated (SpecInt-95).  For a 1 GHz processor clocks,
this results in approximately 120 to 170 clocks of delay
before the data from an average cold access (new DRAM page) 
can be returned to the processor.  This result was largely
insensitive to the choice of L1 and L2 caches latencies.

Wulf et al ~\cite{wulf95hitting} (and Hennessy et al ~\cite{hennpatt95})
report only an average DRAM
latency speed increase of about 7 \%.  
Burger et al ~\cite{burger95declining}
give an average speed increase of between 5 \% and 10 \%.
The slow increase in average DRAM latency is largely due to the
technology used for the core of DRAMs.  The core DRAM technology
is largely the same for all DRAM interface variants.
There was some improvement with the transition to the newer
enhanced SDRAMs that appeared started at the 133 MHz clock
frequencies (for example DDR133).
Processor clocks increase at a much higher rate and approximate
a growth of 40 \% to 80 \% a year ~\cite{davis00new}.
Extrapolating about five years into the future and assuming a
current process clock rate of 2.4 GHz (the latest Pentium 4
processors) and a conservative growth rate of about 40 \%, 
processor clocks
can likely be at around 10 GHz.  However DRAM average access
latencies will only be at about (using the 7 \% figure)
between 100 nsec and 120 nsec.  This represents a latency in
processor clocks of 1000 and 1200 respectively.
%
\section{Background}
%
The need to deal with the processor/memory performance gap has
been reported by many including Wulf
Wulf et al ~\cite{wulf95hitting},
Hennessy et al ~\cite{hennpatt95},
Burger et al ~\cite{burger97limited}, and others.

Cuppu et al ~\cite{cuppu99performance}, Davis et al ~\cite{davis00ddr2},
and Davis et al again later in 2000 ~\cite{davis00new} have
all explored the impact of the newer DRAM interface technologies
on program execution time performance.  
The newer interfaces have also helped with reducing average
access latencies but these are considered one-time fixes to the
overall problem since the core DRAM technology follows the
7 \% growth rate throughout.
%
%
%
\bibliographystyle{latex8}
\bibliography{mem}
\end{document}
%
%
