% executions (for ICS'04)
%
\documentclass[10pt,dvips]{article}
%\documentclass[10pt,twocolumn,dvips]{article}
\usepackage[english]{babel}
\usepackage{epsfig}
%\usepackage{fancyheadings}
%\usepackage[T1]{fontenc}
%\usepackage[latin1]{inputenc}
%\usepackage{twocolumn}
%\usepackage{verbatim,moreverb,doublespace}
%\usepackage{rotate,lscape,dcolumn,array,rotating,latexsym}
%
%\input{epsf}
%
% for somebody (I forget now !)
%\textwidth 175mm
%\textheight 225mm
%\topmargin -4.5mm
%
% for somebody else (I also forget now !)
%\textwidth 6.6in 
%\textheight 239mm
%\topmargin -15mm
%\leftmargin -2.0in
%
% for (IEEE single-column format)
%\textwidth 6.875in
%\textheight 8.875in
%\topmargin -0.6in
%\oddsidemargin 0mm
%\evensidemargin 0mm
%
% for HPCA (IEEE two-column format)
%\textwidth 6.5in
%\textheight 8.875in
%\topmargin -0.4in
%\oddsidemargin 0mm
%\evensidemargin 0mm
%
% for ISPASS and ICS
\textwidth 6.5in
\textheight 8.875in
\topmargin -0.4in
\oddsidemargin 0mm
\evensidemargin 0mm
%
%
% turn the following (linespread) on to 1.6 for "double space"
% \linespread{1.6}
%
%
% some publishers want no page numbers for final print
%\pagestyle{empty}
%
\begin{document}
%
%
%
\title{A Microarchitecture for Parallel Instruction Re-executions}
%
%
\author{
D. Morano, D.R. Kaeli\\
Northeastern University\\
{dmorano, kaeli}@ece.neu.edu
}
%
%
% some publishers do not want a data in the final print
\date{1st March 2004}
%
\maketitle
%
% uncomment the following for first page with no page number (for IEEE)
%\thispagestyle{empty}
%
%
\begin{abstract}
%
We describe a microarchitecture oriented towards
the parallel execution and re-execution of many instructions
simultaneously.
In addition to breaking control dependencies, we also allow
for data dependencies to be broken through the speculative
execution of instructions with predicted source operands.
We have extended the idea of the reservation station, or an
issue window instruction slot, with additional state and logic that
allows for an instruction dispatched to the station to both
issue for execution when suitable source operands are available
but also to re-issue subsequently when further or later source
operands become available.  We call our adaptation of
this enhanced reservation station an \textit{Issue Station}.
Instructions dispatched to these stations remain associated
with the station for the duration of the instruction's lifetime
within the execution window of the machine (until the instruction
is retired being either committed or abandoned).
New instructions are only dispatched to an Issue Station when the
previously dispatched instruction to it is retired.

Rather than determine instruction dependencies at 
instruction issue, we dynamically determine instruction data
dependencies as the instructions are executing (or re-executing).
We employ time ordering tags with both Issue Stations and
all operands for data dependency management within the machine.
By using these time ordering tags, dispatched instructions
are ordered with respect to each other and all active operands
being transferred among machine components.  All operands
are also tagged as they are transferred among the various machine 
components.  All instructions are dispatched in-order but are
allowed to repeatedly issue out-of-order as conditions that
might signal the need for an execution or re-execution occur.
All instructions are retired in-order.

We present preliminary data for the microarchitecture that
focuses on exploring the dynamics of the machine
operation associated with instruction execution, re-execution,
and operand forwarding and acquisition.
The data presented serves to characterize the machine
with varying numbers of different machine components configured.
As expected, the performance of the machine, as measured with
Instructions Per Cycle (IPC), generally increases with increasing
machine resources but also shows where incremental numbers of
components do not appreciably improve performance for given 
configurations.
%
\end{abstract}
%
%
%\vspace{-0.25in}
\section{Preliminary Experimental Results}
%\vspace{-0.15in}
%
Here we present data to characterize the effect of the
number of integer functions units available.
(Other characteristics of the machine configuration
are explored subsequently.)
Memory is idealized in that all memory instructions perform
their operand load or store
function in one clock and with a 100% hit rate on L1 data
cache.  Instruction fetch is also idealized with 100% hit rate on
the L1 instructions cache.  Further, in order to focus more
directly on the dynamics of the instruction execution and re-executions,
we have employed 100% correct branch predictions.
These conditions allow us to focus more intensely on the
machine operations associated with instruction execution and
operand forwarding and acquisition.

Tables \ref{tab:iwsize16a} 
and \ref{tab:iwsize16b} 
shows experimental results
for a machine with a window size of 16 (16 Issue Stations) and various
numbers of integer ALU function units.  
Besides the integer function unit, the only other function units
that are used to any significant extent are the floating add
and floating multiply units, and these are only used significantly 
with the EON benchmark.
Execution is not constrained
by floating point function units
for the benchmarks simulated with the possible exception of EON,
which has a non-trivial amount of floating point instructions
but which is still very small compared with integer instructions.
Dispatch width is set to 256 instructions and thus doesn't 
represent a constraint since we only have 15 Issues Stations configured.

All benchmarks were simulated by first fast-executing the
first 300 million instructions of the program.  This was
followed by a detailed functional simulation of the program
for the next 400 million instructions.
Each of the tables shows the IPC obtained for the benchmark
along with the percentage of all instructions executed that
were re-executions.  
Also given is the percentage of execution
requests that resulted in a stall within a particular Issue Station
when an integer ALU function unit was needed and requested.
These stalls are due to the unavailability of the integer ALU function unit
for a cycle in which an Issue Slot requested one for either an
execution (an initial execution) or a re-execution.
The number of floating point functions units were set high enough
that no significant stalls were ever due to them not being available.

Also note that a stall that may occur by one particular Issue Slot
does not affect other Issue Slots.  All Issue Slots operate
independently from each other and a stall in one does not
directly or necessarily cause a stall in another.
However, if an Issue Slot holds a data dependent instruction
and the anti-dependent instructions gets stalled, and if
a re-execution of the dependent instruction is eventually
needed for proper commitment (because a new and different 
result operand
will be forwarded from the anti-dependent instruction), 
then in effect one Issue Slot
is indirectly causing another to not make progress towards commitment.

Note that for the case where there are 16 integer ALU functions
units configured, there is never a stall due to its unavailability.
This is not unexpected as the total number of instructions executing
simultaneously never exceeds 16 (for this set of results) since
the number of Issue Slots configured were 16 in this experimental set.

As expected, IPC increases for increasing numbers of integer
ALU function units but levels off substantially at eight FUs.
The percentage of executions that are actually re-executions
also increases with increasing integer ALUs but instead appears
to level off close to four FUs configured.


%
\begin{table}[p]
\begin{center}
\caption{{\em Simulation results showing IPC, re-execution percentage,
and percent of stalls due to function unit unavailability.}
These results are for 2 and 4 integer ALU function units.}
\label{tab:iwsize16a}
\vspace{+0.1in}
\begin{tabular}{|l||r|r|r||r|r|r|}
\hline 
{IALU-FUs}& 
\multicolumn{3}{c||}{2} &
\multicolumn{3}{c|}{4} \\
\hline
name &
IPC & \%REXEC & \%WAIT &
IPC & \%REXEC & \%WAIT \\

\hline
bzip2&
2.1&6.3&67.5&	3.5 & 10.0 & 40.0 \\

\hline
crafty&
2.3&6.6&67.3&	3.6 & 10.3 & 40.0 \\

\hline
eon&
3.1&6.8&49.4&	4.2 & 7.9 & 30.0 \\

\hline
gcc&
2.5&5.7&66.7&	4.0 & 8.4 & 40.0 \\

\hline
gzip&
2.1&7.3&69.0&	3.4 & 11.4 & 40.0 \\

\hline
parser&
2.2&6.5&69.2&	3.5 & 9.8 & 40.0 \\

\hline
perlbmk&
2.2&7.3&66.4&	3.7 & 10.4 & 38.4 \\

\hline
twolf&
2.1&9.2&68.2&	3.4 & 12.2 & 40.0 \\

\hline
vortex&
2.4&5.4&65.3&	4.1 & 6.6 & 34.3 \\

\hline
vpr&
2.4&7.7&63.5&	3.7 & 10.9 & 38.3 \\

\hline
\end{tabular}
\end{center}
\end{table}
%


%
\begin{table}[p]
\begin{center}
\caption{{\em Simulation results showing IPC, re-execution percentage,
and percent of stalls due to function unit unavailability.}
These results are for 8 and 16 integer ALU function units.}
\label{tab:iwsize16b}
\vspace{+0.1in}
\begin{tabular}{|l||r|r|r||r|r|r|}
\hline 
{IALU-FUs}& 
\multicolumn{3}{c||}{8} &
\multicolumn{3}{c|}{16} \\
\hline
name &
IPC & \%REXEC & \%WAIT &
IPC & \%REXEC & \%WAIT \\


\hline
bzip2&
5.1 & 12.5 &6.9 & 5.5 & 16.2 & 0.0 \\

\hline
crafty&
5.0 & 14.3 & 7.6 & 5.3 & 17.5 & 0.0 \\

\hline
eon&
4.7 & 9.7 & 2.4 & 4.8 & 10.4 & 0.0 \\

\hline
gcc&
5.3 & 11.1 & 8.5 & 5.7 & 13.4 & 0.0 \\

\hline
gzip&
5.0 & 16.4 & 9.3 & 5.5 & 20.4 & 0.0 \\

\hline
parser&
4.9 & 14.0 & 11.5 & 5.4 & 18.9 & 0.0 \\

\hline
perlbmk&
5.3 & 13.6 & 6.5 & 5.6 & 15.1 & 0.0 \\

\hline
twolf&
4.9 & 15.5 & 8.1 & 5.2 & 18.5 & 0.0 \\

\hline
vortex&
5.5 & 8.5 & 4.4 & 5.6 & 9.6 & 0.0 \\

\hline
vpr&
4.8 & 15.1 & 7.2 & 5.1 & 17.7 & 0.0 \\


\hline
\end{tabular}
\end{center}
\end{table}
%





%
\bibliographystyle{latex8}
\bibliography{executions}
%
\end{document}
%
%

