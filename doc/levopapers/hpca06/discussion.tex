%
%\vspace{-0.25in}
\section{Discussion}
%\vspace{-0.15in}
%
Although the principal aim of this research has been to address
the problem of achieving better execution performance through
greater ILP extraction, it should be noted that the degree to which
faster execution can be achieved can also represent the potential
for an alternate design goal of achieving power savings for
an equivalent execution performance.  Specifically, achieving
a greater IPC can be used for power savings by simple lowing
the clock frequency correspondingly.
Further, we are exploring potential for avoiding
or eliminating redundant re-executions and thereby saving the
power consumption of those re-executions.

With regard to physical scalability, 
the expectation is that larger machines, capable of executing
over a larger instruction window, can provide additional performance
through ILP extraction than that currently possible.
Although other silicon and microarchitecture design tradeoffs (SMT, CMP,
et cetera for example) certainly address a substantial class of 
applications, there will still be important and stubborn serial codes
that cannot be sped up by anything other than more focussed
microarchitectural resources on a single execution thread.
The distributed nature of
our operand management strategy (embodying both dependency determination
and associated routing of actual dependent operand values)
provides substantial potential for implementing much larger
machines (both in terms of physical size and in its numbers of various 
components) than is possible today with the operand management
strategy of conventional superscalars.
More specifically, our proposed microarchitecture allows for
the electrical repetition of the various key buses introducing clock
delays across the silicon that do not present any additional
management or synchronization problems than already handled.
Unfortunately, the possible richness of the idea of buffering 
operands across
electrical domains in the silicon is beyond what can be addressed
presently in this space.
All of this may provide motivation for furthering this line of research.
%
