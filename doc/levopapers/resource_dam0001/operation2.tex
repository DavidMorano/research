
Predicates are snooped and snarfed in a similar manner as data values
but since predicates are chained with a hardware oriented linked-list
scheme, there is no need for the time restriction scheme associated
with snarfing of data values.  Only the predicate addresses being
snooped need be compared, as register addresses were compared with the
forwarded output data snooping.

** continuing **

Note that predicate register addresses are actually time tags of
instructions loaded into some other active stations of the 
instruction window.  For this reason, there must exist
arithmetic decrementing logic needed to substract one from the
column address part of the time tag when a logic left-shift
operation occurs on the instruction window.
As with data dependencies, predicates are chained to
form control dependency trees that correspond to that
in the program being executed.

Just as when a new updated value of a data source becomes
available for an instruction, causing it to become enabled to
execute again, so too can a changed predicate value cause
an instruction to enable to execute again.  A predicate
value can change due to the fact that a previous branch,
for which the instruction is control dependent, either becomes
resolved or changes its prediction for some reason.

If a predicate value is broadcast but its value does not change, any
instructions depending on that predicate rebroadcast their output
values.  This requirement handles the situation where a branch
prediction was changed causing a replacement of a segment of the ML
path with a corresponding DEE path.  In this case, the output values of
the instructions, beyond the branch having the prediction that changed,
have to be rebroadcast since they are usually different than the output
values that were last broadcast from the former ML path of active
stations.  Schemes to selectively avoid this later re-broadcasting of
output values are being worked on but does not affect the general
operation of the machine but merely would be a performance
optimization.

In those cases where an instruction is simultaneously the target of two
or more branches, extra active stations are allocated for the
instruction at instruction load time in order to utilize the extra
cancelling predicate register hardware in the active station.
Figure~\ref{activestation} showed only one set of cancelling predicate
hardware but an implementation may contain more than one set of this
hardware as an optimization.  One set of cancelling predicate hardware
is required for each branch that an instruction may be the target for.
In cases where an instruction is the target of more than one branch and
where there are not enough cancelling predicate hardware sets in the
active station to accommodate the number, the cancelling predicate
hardware sets of the following active station is also used for
detecting when a predicate changes.  In any event, an instruction is
enabled for re-execution when any of its input predicates or cancelling
predicates changed or are just rebroadcast.

Finally, it has been shown that instructions can be speculatively
executed far ahead of the committed execution state while still
being able to eventually re-execute as necessary in order to
eventually correspond to the final committed state.
We have created mechanisms that mange both the data dependencies and
the control dependencies that trigger an instruction to re-execute
when either a predicted data values changes or a predicated
control condition changes.  Using these means, instructions
can be dispatched for execution according to a priority scheme
that only has to minimally consider the availability of resources
in the machine.  Hence, this new paradigm has been termed
resource flow execution.

