
paragraph{Operation: }

The operation of this machine shares many similarities with exiting
machines but also has some key differences.  This section will provide
a more detailed discussion of the dynamic operation of the machine as
it executes a program.  For the purposes of this discussion the initial
boot-up of the machine will be the starting point.  In other words, the
machine is currently empty of any loaded or executing instructions.
First an overview of the machine execution is given followed by a more
detailed description of what occurs within an active station.

In general, operation of the machine will proceed from
instruction-fetch, to branch target tracking, to instruction staging in
the station load buffer.  Next, the entire column of instructions in
the station load buffer will be left shifted into the right-most column
of the active stations in the instruction window (reference Figure 1).
This is termed the {\it load} operation.  The reader may note that the
station load buffer is always the same length as a column in the
instruction window in order for this instruction load operation to
occur as a broadside left shift operation.  These above operations
repeat until the entire instruction window is nominally loaded with
instructions through the left shift-like load operations.  Note that
the authors usually use the term {\it left-shift} for describing the
broadside instruction load operation but this operation can be
accomplished using a renaming scheme on the active station column
addresses in the instruction window.  When the instructions in the
left-most column of the instruction window, those with the smallest
time tags, have all completed executing, they are ready for
retirement.  Retirement may not occur immediately but occurs when the
station load buffer is fully loaded with new instructions ready to be
loaded.  The retirement operation is coincident with a left-shift
operation of all columns in the instruction window along with the
station load buffer column.  This shift removes the retired
instructions from the window and loads the new instructions from the
load buffer into the window, leaving the load buffer empty.

Fetched instructions will also nominally be allocated cache lines in
the I-cache (if the corresponding memory page and physical hardware is
set to allow that).  As instructions are fetched they are then
decoded.  This is, so far, similar to most all current machines.

Once the instructions are decoded, branch type instructions are
identified.  The target addresses of branch instructions are computed,
where possible, and an entry is made in the branch tracking buffer that
includes the time tag of where that branch instruction will be placed
in the instruction window, when the load of it occurs.  The target of
the branch is also placed into the tracking buffer entry.  This
tracking buffer information is used to dynamically track the
instruction domains of instructions within the instruction window.
Instructions are then staged into the active station load buffer.

The station load buffer serves as a staging area to accumulate
instructions until they can enter the instruction window.  When the
instruction load buffer is full and the left-most column of the active
stations in the instruction window are ready to retire, an operation
analogous to a left shift occurs amongst all of the active station
columns and those instructions currently staged in the station load
buffer.  Source registers for loading instructions are taken from the
architected register files (one per instruction window row) at the time
the load occurs.

While instructions are loaded into an active station, they are allowed
to compete for execution, main memory, and the architected register
file resources.  The reader can reference Figure~\ref{shareforward}.
to see the connectivity available to a single sharing group within the
instruction window.  Instructions compete for execution resources
within their sharing groups.  When an instruction sends information to
an execution unit for processing, this is termed {\it instruction
issue}.  Unlike conventional machines, instructions can be issued to
execution or function units many times during the time that it is in
the instruction window.  This will be discussed in more detail later.
Instructions compete for memory bandwidth with all of the other active
stations in the other instruction window columns located in the same
row.  This is illustrated with the contention for the horizontal row
buses to the memory interface buffers shown in Figure ??.  Active
stations also compete to store their output register results into the
architected registers files with all other active stations on the same
instruction window row.  This is illustrated with the contention for
the horizontal row buses to the ISA register files shown in Figure ??.
Speculative reads to main memory are allowed at any time during an
instruction's execution but memory writes are only allowed, in the
current implementation, at instruction retirement time.  Therefor, in
general, instructions can be thought of proceeding through fetch,
decode, branch tracking, staging, load, issue, and
retirement/write-back stages.

As noted already, instructions in an active stations can be issued to
an execution unit more than once.  Instructions are issued to execution
units when one of their inputs is changed but which has a time tag
later than or equal to that of the last source that acquired.  An
active station snoops all forwarding buses originating from earlier
sharing groups for changed input values.  Values are forwarded on these
buses with the address of the architected register that is being
changed along with its value, and along with information indicating its
time tag (time order) within the instruction window.  All forwarding
buses are snooped for updates corresponding to those registers that are
inputs to a current active station.  The register address comparisons
are done using the RI_ADDR register and comparators associated with the
two sources and destination of the active station.

Forwarded branch predicates (representing control flow dependencies)
and forwarded output data values are snooped and processed somewhat
differently.  Due to the physical arrangement of the data forwarding
buses originating from the sharing groups, only values originating from
previous instructions are considered as possible inputs.  However, a
means must used to ignore values coming from previous instructions that
are earlier in time than a previously snarfed data value.  This is
accomplished by using the time tag mask registers (TTMASK) in the
active stations.  Again, there is a time tag mask associated with all
data oriented sources in the active station (including the relay source
value for the instruction's output if there is one).

The time tag mask is actually physically two masks, one representing
that last forwarding span (32 assumed for this discussion) of active
stations and is termed the {\it forwarding bus time tag mask}.  The
other mask represents each of the previous forwarding spans (normally a
column's worth of active stations) and is termed the {\it forwarding
column time tag mask}.  The forwarding bus time tag mask in this
example is 32 bit wide, one bit for each of the 32 forwarding buses
being snooped by the active station.  The column mask in this
implementation example is eight seven wide, one each representing each
of the previous seven forwarding spans (there are eight forwarding
spans in the implementation being described).  Both the forwarding bus
time tag mask and the column time tag mask are ordered corresponding to
the time order of the forwarding buses and previous columns
respectively.  As a convention, we assume that both masks are ordered
such that the right-most bits correspond to active stations (in the
case of the forwarding bus time tag mask) and forwarding span columns
(in the case of the column time tag mask) that are from active stations
earlier in time.  Bits which are set in the masks represent forwarding
buses or forwarding spans that are allowed to be snarfed by this active
station.  Bits which are cleared serve to prevent the snarfing of
data.

If a source input is snarfed from the last forwarding span number of
active stations (within the last 32 active stations in this
implementation), then the position of the bus, as ordered corresponding
to the time of the active station that is originating data on it, is
compared with the forwarding bus time time mask.  If the corresponding
bit in the time tag mask is clear, then no snarfing is performed.  If
the bit in the time tag mask is set, then the data value is snarfed and
all bits to the right of the bit just examined (earlier in time) are
cleared.  The same sort of operation is done analogously with the
column time tag mask when a forwarded value originated from an active
station prior to the preceding forwarding span number of active
stations.  Within each forwarding span of active stations, a generated
output value will never be forwarded beyond a forwarding span (32) if
some instruction within the next 32 active stations also outputs a
value to the same ISA register address.  Since only one output data
value will ever be forwarded beyond a forwarding span of active
stations, this scheme of using a column mask and a forwarding bus mask
ensures that only a value equal to or later in time (but earlier than
the time tag of the snarfing active station) will ever be snarfed.
Additionally, data output are only snarfed if the actual value of the
data has changed from the previously snarfed or held value.  This
latter comparison is done with the equivalence comparators shown in
Figure~\ref{activestation} located along with the RIDAT and ROIDAT
registers.

Predicates are snooped and snarfed in a similar manner as data values
but since predicates are chained with a hardware oriented linked-list
scheme, there is no need for the time restriction scheme associated
with snarfing of data values.  Only the predicate addresses being
snooped need be compared as register addresses were compared with the
forwarded output data snooping.

