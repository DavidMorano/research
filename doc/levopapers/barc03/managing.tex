% managing
%
\documentclass[10pt,dvips]{article}
%\documentclass[10pt,twocolumn]{article}
\usepackage[english]{babel}
\usepackage{epsfig}
%\usepackage{fancyheadings}
%\usepackage[T1]{fontenc}
%\usepackage[latin1]{inputenc}
%\usepackage{twocolumn}
%\usepackage{verbatim,moreverb,doublespace}
%\usepackage{rotate,lscape,dcolumn,array,rotating,latexsym}
%
%\input{epsf}
%
% for somebody (I forget now !)
%\textwidth 175mm
%\textheight 225mm
%\topmargin -4.5mm
%
% for somebody else (I also forget now !)
\textwidth 6.5in
\textheight 8.875in
\topmargin -0.4in
\oddsidemargin 0mm
\evensidemargin 0mm
%
% for (IEEE single-column format)
%\textwidth 6.5in
%\textheight 8.875in
%\topmargin -0.6in
%\oddsidemargin 0mm
%\evensidemargin 0mm
%
% for HPCA (IEEE two-column format)
%\textwidth 6.5in
%\textheight 8.875in
%\topmargin -0.4in
%\oddsidemargin 0mm
%\evensidemargin 0mm
%
%
% some publishers want no page numbers for final print
%\pagestyle{empty}
%
\begin{document}
%
% going to 1mm here recoups some good space ! :-)
%\parskip 1mm
%\parskip 2mm
%
%
\title{Managing Control and Data Dependence in Future Out-of-order
Microarchitectures}
%
\author{
D. Morano, D.R. Kaeli\\
Northeastern University\\
{dmorano, kaeli}@ece.neu.edu\\
\and
A.K. Uht\\
University of Rhode Island\\ 
uht@ele.uri.edu
}
%
%
% some publishers do not want a data in the final print
%\date{}
\date{11th December 2002}
%
\maketitle
%
% uncomment the following for paper with no page numbers (for IEEE)
%\thispagestyle{empty}
%
%
%
%
Current high performance superscalar processors have widely employed
instruction level parallelism (ILP) to achieve their high single-threaded
performance levels.
The primary mechanisms used in these microarchitectures for implementing
their ILP extraction have been the issue window,
multiple function units, 
reorder buffer,
and the architected register file.
Instruction windows can presently handle about 128 instructions
and larger instruction windows are still likely.
Current microarchitectures also widely use speculative instruction
execution through the prediction of future control flow.
This speculative execution is primarily done by predicting the outcomes
of individual condition branches and then speculatively executing instructions
on the predicted control flow path.  Although successive conditional
branches can be and are also predicted, speculative execution essentially only
proceeds further on a single speculative control-flow path.
Proposals have been made to execute along more than one speculative
control-flow path simultaneously (multipath execution)
but several proposed techniques for doing this incur
high additional cost in control logic.
Either additional entire reorder buffers are introduced for the alternative
speculative paths or some other additional specialized control logic
needs to be introduced.
As a result, multipath execution has not as yet been considered to be cost 
effective to do in present processors.
There have also been proposals to predict data values for program
variables.  These include not just branch target addresses or memory operand
addresses, but also register and memory operand data values.
Although this approach seems promising, it adds a substantial additional 
logic
burden to the control portion of any microarchitecture that employs
the technique.
Finally, there are proposals to facilitate the execution of
control-independent instructions beyond unresolved conditional branches.
This technique is also generally expensive in terms of control logic
needed to properly coordinate.

Another complication for existing and future processors is the large
integrated circuit routing congestion for 
access to their centralized resources.
These centralized resources generally include the issue window,
the architected register file, the reorder buffer, and possibly
the load-store queue for memory operands.
Further, as the number of instructions in flight increases, the need
for access to the centralized resources of the microarchitecture
increases proportionally and contention for limited ports (on any
of the centralized resources named above) begins to impose diminishing
performance returns through the forced waiting for resource 
read or write port access.
As a result of these problems, present processors
are rapidly nearing the limits of what is possible with the exploitation
of ILP through larger instruction windows.

Because of many of the present and foreseeable problems associated
with elaborate control logic for dynamically scheduled superscalar processors,
alternatives such as explicit parallel instruction architectures
or multithreaded microarchitectures are being aggressively explored.
However, neither of these approaches has yet been proven to substantially
increase
execution performance of a single thread of control, which is still
an important goal for high performance computing.
In spite of the present problems associated with instruction and operand
dependence management for dynamically scheduled processors, 
it is still advantageous and desirable
to explore ways to combine many of the promising microarchitectural
techniques mentioned already, such as data value prediction,and speculative
multipath execution, and execution of control-independent instructions.
We present a new approach for the simultaneous management of speculative
instruction execution, speculative multipath execution, and for
the ordering of speculative and non-speculative operands.
Our approach both unifies the handling of control-flow dependencies
and data dependencies as well as reducing the amount of control
logic needed for large or very large instruction window sizes.

Our approach to program dependency ordering (both control-flow
and data-flow) is realized by tagging all instructions that
enter into consideration for execution (non-speculative or otherwise)
with tags that reflect the dynamic program order.
These tags essentially store a time stamp of a sort to place the
instruction into its proper dynamic order as compared with all other
instructions being executed.  We term these tags \textit{time-tags}
in light of the way that they are used.
We also dynamically predicate all instructions, within the microarchitecture
itself, to allow for large amounts of possibly random out-of-order
speculative execution.
This feature allows for the benefits of predicated execution to be
applied to those processor
architectures that do not have architecturally visible predication.
This technique is still applied even for processor instruction
sets that do have predicates also.  This proposal is thus independent
of any compiler changes and allows for execution of all legacy program
codes without recompilation.
We further tag all program operands, whether they be register operands,
memory operands, or the dynamically introduced microarchitectural
predicate operands that are now associated with all instructions.
For those architectures that may already have architected predicate
registers, these are just registers as far as our technique is concerned
and are simply treated as such without loss of any generality.
Our use of dynamic predication also allows for an easy way (low hardware
cost) to
govern the execution of control-independent instructions.

Time-tags are just small integers that have a number of bits that is
approximately log-base-two the number of instructions allowed
in flight in any given machine implementation.
The value of a time tag is incremented for each successive instruction
that enters into execution consideration.  
Generally, that
instruction that is oldest in program ordered time and nearest to
program commitment is assigned
a time-tag value at or near zero, while those instructions (both
non-speculative and speculative) that are younger and more speculative
are assigned successively higher values.
These tags are periodically
uniformly decremented throughout the whole of the execution core
in order to prevent overflow related issues.  
In our scheme, all instructions that have entered into consideration
for execution remain in a microarchitectural structure that resembles
an issue slot or a reservation station until the instruction is
retired (either squashed or committed).

To allow for value prediction, instructions are allowed to
issue to function units repeatedly as needed or desired until they are
retired.  Upon instruction issue, although a function unit is
allocated for execution, the instruction also stays in its
assigned reservation station.
To facilitate multipath execution, a path identifier is also
assigned to each instruction that enters into execution
consideration.  This is needed since more than one
instruction can have the same relative position
in program ordered time (as compared with the last committed
instruction) but still have to be disambiguated since they reside on 
different speculative paths of control.
The combination of the architected name for an operand (either
its architected register address or its memory address) together
with the path identifier and the time-tag form a unique microarchitectural
name for the operand.  Thus the microarchitecture essentially provides
for the full renaming of all architected operands.

Once all instructions and operands are tagged using our technique,
all instruction are allowed to issue and execute in any order
and with any input operands regardless of whether the input operands
are correct or not.  In those circumstances when an input operand
is not the final committed program value, the instruction will
dynamically figure out through operand snooping that it will need
to re-execute at least one additional time (with the correct
committed input operand value) before it can be committed (if it is
even going to be committed).  Instructions that are both on
the eventual committed program path as well as those that are
on alternative speculative paths are allowed to execute simultaneously.
Input operands to instructions later in program-ordered time snoop
for output operands from previous instructions.
Executions or re-execution only need occur when input operands
change from previous values.
Eventually, all correct operands are snarfed and if the oldest
dispatched instructions have executed at least once, commitment of
those instructions can occur.
This amount of flexibility allows for increased levels of instruction
parallelism since more parallelism is being exposed from the
program itself.

We present the simulated results of a representative microarchicture
using the ordering technique discussed.
SpecINT benchmarks were compiled using a standard vendor compiler and
then run on the simulator using a current representative memory hierarchy
and memory access latencies.  Instructions per clock (IPCs) numbers for
each benchmark is provided for varying machine configurations and
range from about 3.8 to 5.2.
%
%\bibliographystyle{latex8}
%\bibliography{managing}
%
\end{document}
%
%
%
