%
\subsection{Operand Forwarding Strategies and Bus Transactions}
%
Although we have so far described the operand forwarding mechanism
in simple terms as being the broadcasting of operands
to those instructions with higher valued 
time tags, there are some additional details that need to be
addressed for a correctly working forwarding solution.
These details also differ depending on type of operand.
There are many possible strategies for forwarding of operands
(and of operands of differing types). 
We now briefly outline three such strategies.
One of these is suitable for registers.
Another is suitable for registers or memory operands.
The third is oriented for the forwarding of predicates.
These three strategies are termed \textit{relay forwarding},
\textit{nullify forwarding}, and \textit{predicate forwarding}
respectively.
In general, each forwarding strategy employs bus transactions
of one or more types to implement its complete forwarding solution.
%
%
\subsubsection{Relay Forwarding}
%
This forwarding strategy is quite simple but is also entirely
adequate for the forwarding of register operands.
In this strategy, when a new register operand needs to be forwarded
from an instruction, the standard operand information, as 
previously described
in general, is packaged up into what is termed
a \textit{register store} transaction.
This transaction type consists of :
%
\begin{itemize}
\vspace{-0.10in}
\item{transaction ID of \textit{register-store}}
\vspace{-0.10in}
\item{path ID}
\vspace{-0.10in}
\item{time tag of the originating instruction}
\vspace{-0.10in}
\item{register address}
\vspace{-0.10in}
\item{value of the register}
\vspace{-0.10in}
\end{itemize}   
%
A request is made to arbitrate for an outgoing forwarding bus
and this transaction is placed on the bus when it becomes available.

When an instruction obtains 
a new input operand, it will re-execute, producing a new
output operand.  In this forwarding strategy, the new output
operand is both stored locally 
and is sent out on the outgoing forwarding buses
to subsequent (higher time-tag valued) instructions.
Previous values of the instruction's output operand are also
snooped as if they were input operands and are also stored locally.
It should also be noted that if the enabling execution predicate
for the current instruction changes, either from being enabled
to disabled or visa versa, a new output operand is forwarded.
If the instruction predicate changes from disabled to enabled,
the output operand that was computed by the instruction is
forwarded.  If the instruction predicate changes from enabled
to disabled, the previous value of the output operand (before being
changed due to the instruction execution) is forwarded.
That previous value is available to the instruction because
it gets snooped as if it was an additional input.
Newly forwarded operands will always supersede any previously
forwarded operands.
With this strategy, instructions that are located in the program
ordered future will eventually always get the correct
value that will end up being the committed value if the
current instruction ends up being committed itself (ends
up being predicated to execute).
This is an elegant forwarding strategy and
the simplest of the forwarding strategies investigated so far, and
is a reasonable choice for the handling of register operands.
The inclusion of the time tag in the transaction is the
key element that allows for the correct ordering of
dependencies in the committed program.
%
%
\subsubsection{Nullify Forwarding}
%
There are limitations to the applicability of the previously
discussed forwarding strategy (relay forwarding).
That strategy depends upon the fact that the address of the
architected operand does not change during the life time of
the instruction while it is executing.
For example, the architected addresses for register operands
do not change for instructions.  If the instruction takes
as an input operand a register \textit{r6}, for example,
the address of the operand never changes for this particular
instruction (it stays \textit{6}).
This property is not generally true of memory operands.
The difficulty with memory operands is that many memory
related instructions determine the address of a memory operand
value from an input register operand of the same instruction.
Since we allow for instructions to execute and re-execute
on entirely speculative operand values, the values of
input register operands can be of essentially any value
(including a wildly incorrect value) and thus the
address of a memory operand can also change while 
the instruction is in flight.
This presents a problem for the correct enforcement of
memory operand values and the dependencies among them.
If we examine the case of a memory store instruction,
when it
re-executes acquiring a new memory store value, the address of that
memory store may also have changed !  
We cannot simply forward that new memory operand (address and value)
as with the relay forwarding strategy above.  The reason is
that we would not be superseding the previous memory operand
that we forwarded previously because it quite likely had a different
architected address.  Rather, we need some way to cancel the effect of
any previously forwarded memory operands.
This present forwarding strategy does just that.

In this strategy, memory operands that need to be forwarded
employ a similar transaction as above for registers (described
in the context of relay forwarding) but would instead have
a transaction ID of \textit{memory-store} and would
include the memory operand address and its value (along with the
path and time-tag information).
However, when an instruction either re-executes or
its enabling predicate changes to being disabled, a different
type of forwarding transaction is sent out.
This new type of transaction is termed a \textit{nullify transaction}
and has the property of nullifying the effect of a previous
store transaction to the same architected operand address.
This transaction type consists of :
%
\begin{itemize}
\vspace{-0.10in}
\item{transaction ID of \textit{memory-nullify}}
\vspace{-0.10in}
\item{path ID}
\vspace{-0.10in}
\item{time tag of the originating instruction}
\vspace{-0.10in}
\item{memory operand address}
\vspace{-0.10in}
\item{value of the memory operand}
\vspace{-0.10in}
\end{itemize}   
%
When this transaction is snooped by subsequent instructions,
for those instructions that have a memory operand as an input
(that would be for instructions that load memory values in
one way or another),
a search is made for a match of an existing memory
operand.  If a match is detected,
the time-tag of that particular memory operand is set to
a state such that any future \textit{memory store} transaction,
regardless of its time-tag value, will be accepted.
Further, on reception of this \textit{memory nullify} transaction,
a request is sent backwards in program order for a memory
operand with the desired memory address.
The transaction that represents a request for a memory
operand would consist of :
%
\begin{itemize}
\vspace{-0.10in}
\item{transaction ID of \textit{memory-request}}
\vspace{-0.10in}
\item{path ID}
\vspace{-0.10in}
\item{time tag of the originating instruction}
\vspace{-0.10in}
\item{memory operand address}
\vspace{-0.10in}
\end{itemize}   
%
Of course, the memory address for the operand desired
needs to be in the transaction, but it is not as obvious why
the originating instruction's time tag is also included.  In some
interconnection fabrics, the time tag is included in backwarding
requests to limit the scope of the travel of the transaction
through the execution window.  This same scope-limiting function
is usually performed for forward going transactions as well.
When the request is sent backwards in program order, previous
instructions or the memory system itself will eventually snoop
the request and respond with another \textit{memory store}
transaction.
As discussed, this forwarding strategy is very useful for memory
operands but it can also be used for register operands with
appropriate changes to the applicable transaction elements.
Again, the inclusion of a time tag value is what allows
for proper operand dependency enforcement
in the committed program.
%
%
\subsubsection{Predicate Forwarding}
%
There are several ways in which instructions can be predicated
in the microarchitecture.  
Some of these techniques are discussed in the papers by
by Uht et al ~\cite{Uht01} and Morano ~\cite{Morano02}.
Predicate register values are essentially
operands that need to be computed, evaluated, and forwarded
much like register or memory operands.
Each instruction computes its own enabling predicate by
snooping for and snarfing predicate operands that are forwarded
to it from previous instructions from the program-ordered past.
Depending on the particular predication mechanism used,
relay forwarding (described above) may be a suitable (if not a good) choice 
for handling the forwarding of predicate operands.
However, some predication mechanisms need additional transaction
types (besides a base store transaction) to communicate.
For example, the predication strategy described by Morano ~\cite{Morano02}
requires three transactions to be fully implemented and is
used as the example in this section for discussing the forwarding
of predicate information.

This predication strategy requires two store-type transactions
rather than just one (as with register or memory operands).  
These two transactions are similar
to the previous operand store transactions 
but one of these holds two values rather than just one.
The first of these is the \textit{region predicate store}
transaction and consists of :
%
\begin{itemize}
\vspace{-0.10in}
\item{transaction ID of the \textit{region-predicate-store}}
\vspace{-0.10in}
\item{path ID}
\vspace{-0.10in}
\item{time tag of the originating instruction}
\vspace{-0.10in}
\item{region predicate value}
\vspace{-0.10in}
\end{itemize}   
%
This transaction is analogous to a register or memory
store, but instead is used to forward a single bit value (the
current \textit{region predicate} for instructions following the
instruction that forwarded the transaction).  A region predicate
is a single bit that determines the execution status
(enabled or disabled) for instructions that lie beyond the
not-taken output path of a conditional branch.
This particular transaction could be forwarded by either
a conditional branch or by a non-branch instruction.
In the
case of a non-branch instruction, the only
predicate value that makes sense is the same as its
own enabling predicate, and so only one value needs
to be forwarded.

In the case of a conditional branch instruction,
there are two possible output predicates that need to be considered: 
1) for the taken output path and
and 2) for the not-taken path.
In order, to forward both values for these instructions,
to program-ordered future, the second store transaction
type (mentioned previously) is used.
This transaction, termed a \textit{branch target predicate store},
consists of :
%
\begin{itemize}
\vspace{-0.10in}
\item{transaction ID of \textit{branch-target-predicate-store}}
\vspace{-0.10in}
\item{path ID}
\vspace{-0.10in}
\item{time tag of the originating instruction}
\vspace{-0.10in}
\item{branch target instruction address}
\vspace{-0.10in}
\item{region predicate value}
\vspace{-0.10in}
\item{branch target predicate value}
\vspace{-0.10in}
\end{itemize}   
%
This is identical to the previous \textit{region predicate store}
transaction but also includes the instruction address
for the target of the conditional branch (the \textit{taken} address)
and the single bit predicate
governing the execution status for instructions
following the target of the conditional branch in program-ordered
future.

Finally, a third transaction is used to invalidate a previously
forwarded branch target predicate.  This transaction is
a \textit{branch target invalidation} and consists of :
%
\begin{itemize}
\vspace{-0.10in}
\item{transaction ID of \textit{branch-target-invalidation}}
\vspace{-0.10in}
\item{path ID}
\vspace{-0.10in}
\item{time tag of the originating instruction}
\vspace{-0.10in}
\item{branch target instruction address}
\vspace{-0.10in}
\item{time tag of target predicate to be invalidated}
\vspace{-0.10in}
\end{itemize}   
%
This is similar to other such invalidation transactions in
that when it is snooped by instructions in the program-ordered future,
a search is made for some state (in this case some predicate
register state) that matches the given transaction criteria.
The inclusion of the second time tag in this transaction allows
for certain efficiencies that are particular to the predication
mechanism described.
%
%
