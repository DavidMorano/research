%
\subsection{Multipath Results and Discussion}
%
In this section we present data corresponding to varying the
maximum number of alternative speculative paths that are allowed.
Figure \ref{fig:figall} shows the speedup results
when multipath execution is enabled.  Results for
each benchmark program is presented.  The results for each benchmark
consists of six groups where each group represents the results
for one of six machine configurations.

Speedups with each group of results is relative to the
single-path case with no alternative speculative paths spawned
for any conditional branches.  For each benchmark program and
for each of the six machine configurations explored, speedups
for cases with a maximum
of zero (leftmost) to seven (rightmost) additional alternative
paths are allowed.

The most speedup gained occurs for the {\tt go} benchmark.  
To explain this, we need
to note that this benchmark has the lowest branch prediction rate
of those that we executed.
Spawning disjoint alternative paths has the effect of reducing the
branch misprediction penalty.

The {\tt bzip2} benchmark has the lowest speedup of the group.  
If we look at Figure \ref{fig:numbranches} we 
see that branches for the {\tt bzip2} program have the
largest branch domain size with respect to other benchmarks and as a
result there is less opportunity for spawning disjoint paths.

We also observed a significant speedup for the 8-8-16 machine configuration
of for the {\tt gap} program.  
Our preliminary investigations suggest
that we might have captured
a loop in our execution window that makes for a substantial speedup 
through the elimination
of the misprediction penalties of its branches.
