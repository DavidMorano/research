% multipath execution on a large distributed microarchitecture
%
%\documentclass[10pt,twocolumn,dvips]{article}
\documentclass[10pt,dvips]{article}
\usepackage[english]{babel}
\usepackage{epsfig}
%
%\usepackage{fancyheadings}
%\usepackage[T1]{fontenc}
%\usepackage[latin1]{inputenc}
%
%\usepackage{twocolumn}
%\usepackage{verbatim,moreverb,doublespace}
%\usepackage{rotate,lscape,dcolumn,array,rotating,latexsym}
%
%\input{epsf}
%
\textwidth 6.5in
\textheight 9.0in
\topmargin -0.6in
\oddsidemargin 0mm
\evensidemargin 0mm
%
%\pagestyle{empty}
%
\begin{document}
\parskip 2mm
%
%
\title{Improved Value Speculation}
%
\author{
D. Morano\\
Northeastern University\\
dmorano@ece.neu.edu\\
}
%
\maketitle
%\thispagestyle{empty}
%
% abstract goes here if you have one
%
%
\section{Introduction}
%
This document will outline an improved method of handling
value speculation in Levo-like microarchitectures.
The improvement is a relatively small change on the current
method and only presents a small increase of hardware in
each Active Station (AS).
However, the improvement may be useful when either Processing Element (PE)
or Forward Bus bandwidth needs to be conserved or reduced.
The improvement reduces the amount of speculative execution that
results from speculated values that originate in the current scheme,
while at the same time not directly contributing to longer residency
times (in clocks) for instructions within the execution window 
of the machine.
%
\section{Current Value Speculation Technique}
%
The current value speculation scheme consists of slightly different
actions depending on the state of an executed instruction that
is loaded into an AS.
When an instruction gets dispatched (transferred from a fetch buffer
to an AS) it enters the execution window and is ready to execute
if resources are available.  
The first thing that an instruction
needs before requesting an execution resource is to determine
its source operands.  For the MIPS-1 ISA, up to five 32-bit source operands
may be required for any single instruction.
Note that for relay forwarding (one of several possible forwarding
strategies), initial destination operands also have to eventually be
acquired.  But this is not a constraint on when as AS can
first make an execution request.
After receiving initial source operands,
ASes then receive additional operands (forwarded from previous
ASes) during its residency in the execution window.
This represents two different phases for getting source operands :
initial operands, and subsequent operands.  Each of these phases
is discussed next.
%
\subsection{Getting Initial Source Operands}
%
Three alternatives have been proposed so far for acquiring
initial source operands for an AS.  Each would seem to have
some merit but none has so far distinguished itself above the others
so far as being better.

The first alternative is for the AS to request each of its operands
(source or destination) by making requests for them on Register
Backwarding buses.  The AS then waits for the operands to arrive.  Once
all operands are present in the AS, the AS is free to request execution
resources.

In the second alternative, a hybrid approach toward getting source
operands is used.  No backwarding request is made for the most common
source operands (usually the orthogonal integer registers), but
requests are made for the more rare operands such as a status register
for example or an FP register (whether an FP register is considered
rare or not might be debatable but that is what was done in this scheme
so far).  When a new column of ASes is first loaded, this signals the
register filtering unit just before the column (in time-tag ordering)
to forward all of the "common" registers on forwarding buses.
The operands that are requested also get forwarded as the requests
are honored by either register filtering units or other ASes (time-tag
ordered before the newly loaded column).  
Note that in this scheme, some registers may be forwarded for which
no newly loaded ASes require values for.  This represents a waste
of bus bandwidth but would not present a problem when more bandwidth
is provided than is needed.  The hope in this scheme is that
needed initial operands will arrive sooner at an AS since the
process of requesting the most usual (common) operands was eliminated.

In the third alternative, initial values are not requested and neither
are values of common registers forwarded automatically.  Rather,
initial values are guessed using a value predictor.  This scheme is
much more costly in terms of hardware resources since a value
predictor represents a sizable amount of real estate.  Further,
due to the need for parallel dispatch of instructions to ASes,
typically a separate value predictor is required for each row of
the execution window (ouch).

Of course, combinations of the above techniques are possible.
%
\subsection{Getting Subsequent Source Operands}
%
Once an AS has its initial source operands, it competes with
other ASes for execution resources.  Once an AS gets to execute,
it forwards its result operands.  Up to three 32-bit result
operands are possible in the MIPS-1 ISA.
Of course, ASes in program ordered
past (lower valued time-tags) may have also forwarded result
operands and these are being snooped by the AS currently in view.
In the present scheme, once an AS snarfs any one of its
source operands from previous ASes, it initiates a request
for execution again.  If an AS currently requires five
source operands (for example), a newly snarfed value for any
one of them triggers the present AS to re-execute its instruction.
Of course, this only occurs when the value snooped is different
than the previous value held by the AS.
%
\subsection{Problems With the Present Scheme}
%
The present scheme has the merit of being simple (as far as
any of this might be considered simple in the first place)
and also does not impose any direct delays for requests of
execution or for generating new result operands.
However, there are some things about this scheme that are
not as attractive as they could be.
For those cases where two or more input operands get snarfed
by an AS, but in different clock periods, one or more needless
executions of the instruction in the AS might have occurred.
Since only the results from the latest execution are retained,
the results from the earlier execution might be considered
wasted.  In this case, we can think of one unit of execution
resource as having been wasted.  However, there is more wasted
than just this execution resource.  The result operands from both executions
are forwarded to subsequent ASes.  The result operands from the
first execution will trigger executions for subsequent ASes
that have a dependency (which may be real or not) on the forwarded
operand.  This forms a sort of 
chain reaction (following the dependency chain),
but this is 
what is intended to maintain proper program order.
The problem is that any subsequent executions based on the forwarding
of the first execution results of the present AS, may also be wasted.
This is so since those same subsequent executions by succeeding ASes
in the dependency chain will likely
execute again when the operands from the present AS's second
execution gets forwarded.
This represents both additionally wasted execution resources
and wasted forwarding bus bandwidth.
In the new scheme being proposed, some waste due to needless
executions, can be eliminated.
%
\section{Improved Speculation Scheme}
%
The new scheme is the same as the present scheme with respect
to how initial source operands are acquired by ASes.
Any of the three methods discussed, as well as others, may be
employed.  The new scheme differs slightly from the present
scheme in the handling of snarfed operands
subsequent to acquiring the initial operands.
In this scheme, a small counter is maintained by each AS.
The counter is clocked by the system clock.
Upon snarfing an operand (its value was different than the
current value), the counter is enabled to start counting.
The terminal count for the counter should be some number of
system clocks that is more than about two and significantly less than
the average latency that instructions stay within the execution
window.  A good number may be about the average
clocks between column shifts, plus or minus one half this or double this
number.  

No execution request is made by the AS immediately after snarfing an operand.
Rather, a certain number of clocks (as maintained by the counter) 
are allowed to elapse.  If the counter reaches its termination count,
the AS then initiates its execution request.  If a new operand is
snarfed during the time the AS is waiting for counter expiration,
the AS initiates its execution request immediately, using the newer operand,
without waiting for counter expiration.  In addition, ASes that find
themselves in the column that is in the position to commit next
do not follow this above rule with the counter but instead just
initiates execution requests upon the arrival of each and every
newly snarfed operand.  This is done so as to not add any direct
dead clock periods that can be attributed to a delay in the
next column commitment.

Using this strategy, up to one extra operand, that is snarfed by an AS
within the period of the counter duration, can be eliminated.
For each snarfed operand that is eliminated, its
redundant execution is also eliminated along with any of its
redundant result operands.  Of course, for chains of data
dependent instructions, just as a chain reaction of redundant
executions and forwards was created by each additionally snarfed
source operand, that same chain reaction of operand forwards
and executions will also be eliminated.

The idea with this strategy is to reduce contention pressure on
execution resources and forwarding bus resources while not
adversely affecting the residency time of instructions in the
execution window.  
By not applying the counter
expiration waiting rule for initially loaded ASes and ASes 
located in the next column that is schedule to commit, no dead
clocks are directly added to the overall instruction residency
time.
%
\section{Other Possibilities}
%
Other choices for how as AS waits for additional operands
are possible.  One such approach could be for an AS to always
wait for counter expiration rather then initiating its
execution request on the snarf of an operand within the counter
period.  This would offer the possibility of eliminating more
than just one redundantly snarfed source operand and its associated
execution chain reaction.
%
\section{Conclusions}
%
A a new scheme has been introduced that can be used to reduce
the requirement for execution resources and forwarding bus
bandwidth.  This is accomplished through the elimination of
certain redundant snarfs and associated re-executions by an AS.
The new scheme does not add any clock delays to the amount
of execution window residency until possible commitment.
This new scheme may be useful for machine configurations may have
a high AS to PE ratio where there is high contention for execution
resources by the ASes.  This new scheme may also be used to
reduce the demand and contention for forwarding bus bandwidth.
Where there is sufficient execution resources and forwarding bus
bandwidth, this scheme should perform no worse than the previous
speculative value handling method.
%
%
%
\bibliographystyle{latex8}
\bibliography{pred}
\end{document}
%
%
