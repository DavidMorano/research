% Out-of-order
%
%\documentclass[10pt,dvips]{article}
\documentclass[10pt,twocolumn,dvips]{article}
\usepackage[english]{babel}
\usepackage{epsfig}
%\usepackage{fancyheadings}
%\usepackage[T1]{fontenc}
%\usepackage[latin1]{inputenc}
%\usepackage{twocolumn}
%\usepackage{verbatim,moreverb,doublespace}
%\usepackage{rotate,lscape,dcolumn,array,rotating,latexsym}
%
%\input{epsf}
%
% for somebody (I forget now !)
%\textwidth 175mm
%\textheight 225mm
%\topmargin -4.5mm
%
% for somebody else (I also forget now !)
%\textwidth 6.5in
%\textheight 8.875in
%\topmargin -0.4in
%\oddsidemargin 0mm
%\evensidemargin 0mm
%
% for (IEEE single-column format)
%\textwidth 6.5in
%\textheight 8.875in
%\topmargin -0.6in
%\oddsidemargin 0mm
%\evensidemargin 0mm
%
% for HPCA (IEEE two-column format)
\textwidth 6.5in
\textheight 8.875in
\topmargin -0.4in
\oddsidemargin 0mm
\evensidemargin 0mm
%
%
% some publishers want no page numbers for final print
%\pagestyle{empty}
%
\begin{document}
%
% going to 1mm here recoups some good space ! :-)
%\parskip 1mm
%\parskip 2mm
%
%
\title{Exploring Parallel Out-of-order Re-execution}
%
\author{
D. Morano, D.R. Kaeli\\
Northeastern University\\
{dmorano, kaeli}@ece.neu.edu
}
%
%
% some publishers do not want a data in the final print
%\date{}
\date{8th January 2004}
%
\maketitle
%
% uncomment the following for paper with no page numbers (for IEEE)
%\thispagestyle{empty}
%
%
%
Current high performance superscalar processors have widely employed
instruction level parallelism (ILP) to achieve their high
single-threaded performance levels.  
The primary mechanisms used in
these microarchitectures for implementing their ILP extraction have
been
control flow prediction,
out-of- order instruction issue and speculative execution, multiple parallel
function units, and the use of a reorder buffer for dependency management
and for instruction squash.
Large instruction windows have been important for ILP extraction
and
instruction windows of 
from about 100 to 200 instructions are in use with larger ones still
likely.  

Although control flow prediction and speculative execution
have been paramount towards improving ILP extraction from 
serial programs, various proposals for
the use of value prediction have also appeared to offer promise
of even more ILP extraction.
Many value prediction proposals have focused on memory load
addresses or memory load values.
A more aggressive value prediction proposal would include not just
the prediction of branch target addresses (direct or indirect)
or memory operand addresses (and in some cases memory operand values
-- like the Pentium-4 microarchitecture), 
but also register
operand data values -- indeed all operand values.  
Although value prediction seems
to have been promising, 
using it on a large scale adds a substantial 
additional logic burden to the control
portion of any microarchitecture that employs the technique.  
The existing complexity of implementing value prediction has
prevented its investigation into larger and more parallel 
microarchitectural designs using it.

We present a microarchitecture that features the ability to
re-execute 
instructions as a primary and expected event.
In our microarchitecture, instructions are dispatched in decoded form
to microarchitectural structures that closely resemble
reservation stations or instruction issue window slots.
However, rather than having the instruction vacate the issue station
when it is transferred to an execution pipeline,
in our microarchitecture the instruction remains in the issue station
structure until it is retired (either being committed or abandoned).
The instruction will request an execution resource when needed
but will wait for result operands from execution to return to 
the station structure rather than proceeding to update something
like a reorder buffer or the register file (either architected or
speculative).
The instruction station structure also holds all input operands
and output operands for the instruction along with other state
information that guides the instruction through repeated
executions until it is retired.
Generally instructions will request execution when all of their
input operands become available whether they may be speculative or not
(similar to the classic reservation station)
but they can also request additional re-executions when some
condition changes that warrants or advocates a new execution.
Conditions that indicate the need for a re-execution include
a new input operand arriving from a previous instruction (which
may have also re-executed providing a new result), 
or a new input operand prediction
being made.

Using this microarchitectural approach, instructions can be 
executed and
re-executed with a low overhead.  This low overhead facilitates
large-scale parallel use of value prediction for input operands
that has not yet been well explored in other proposed microarchitectures.
Instructions that re-execute
do not have to reflow through any of the: fetch, decode, rename, or dispatch
stages of the machine.  Although this approach is not entirely 
new (witness Pentium-4 with
load instructions) we have elevated the process of speculative re-execution to
a level not seen in any conventional microarchitecture or proposed
microarchitectures.

Our approach to program dependency ordering 
is realized by tagging all instructions that get
dispatched to issue stations with tags
that reflect the dynamic program order.  These tags essentially 
represent a
time stamp of a sort to place the instruction into its proper dynamic
program order as compared with all other instructions being executed.  
We further tag all program operands, whether
they be register operands or memory operands.
The program ordering tags are just small integers that 
have a number of bits that is
approximately log-base-two the number of instructions allowed in flight
in any given machine implementation.  The value of a tag is
that is assigned to a dispatched instruction 
(effectively assigned to the issue slot itself)
is successively higher for each new instruction.
Generally, the instruction that is oldest in program
ordered time and nearest to program commitment is assigned a tag
value at or near zero, while those instructions (both non-speculative
and speculative) that are younger and more speculative are assigned
successively higher values.  These tags are periodically uniformly
decremented, at the occasion of instruction commitment,
throughout the whole of the execution core in order to
prevent overflow related issues.  
Although a register file is used for storage of the architected
registers, no renaming registers are needed in this microarchitecture.
All instruction operands are effectively automatically and
uniquely named by the association of the operand with its program
ordering tag.  In this way, all false operand dependencies are
eliminated without having to provide explicit register renaming
resources, the number of which could have been otherwise limiting for
instruction issue if there was not
a sufficient number available as compared with the number of operands in
flight in the execution core.

The execution resources of this microarchitecture are similar
to other machines.  We employ several pipelined function units
of varying pipeline depth depending on the execution latency for
the different units.  Unlike conventional machines, execution results
are returned to the issue station from which the originating
decoded instruction operation came from.  
Use of a particular function unit is
available to any set or group (possibly including all) of the 
instruction issue stations.  Of course, since each issue station
may be occupied by any type of instruction, at least one of each
type of function unit is available to each issue station for use,
but different forms of function-unit multiplexing can be explored.

Although all instructions are dispatched to issue stations in-order,
once all instructions and operands are tagged using our technique, all
instructions are allowed to issue and execute in any order and with any
suitable input operands regardless of whether 
the input operands are correct or not.  
In those circumstances when an input operand is not the final
committed program value, the instruction will dynamically determine
through operand snooping that it will need to re-execute at least one
additional time (with the correct committed input operand value) before
it can be committed (if it is even going to be committed).
All instructions are allowed to execute simultaneously as available
execution function unit resources permit.  
Input operands to instructions later in
program-ordered time snoop for output operands from previous
instructions.  In this way, all dependencies are determined
dynamically and in a distributed manner.
This distributed operand dependency mechanism facilitates
easier expansion of the machine size (numbers of various units) 
while still allowing for the proper control to achieve the
maximum amount value prediction, all in parallel.
Executions or re-executions only need occur when input
operand values change from their previous values.  
Eventually, all correct operands
are snarfed and if the oldest dispatched instructions have executed at
least once, commitment of those instructions can occur.  
Any number of instructions (at the granularity of a single
instruction -- unlike some other microarchitectures) 
is allowed to commit in each clock period.
This microarchitecture is also suitable for use in implementing
any of the existing instruction sets or any foreseeable new
ones also (including those that have architected predication).
Using this new microarchitecture with current sized (around 200 instructions)
or larger instruction windows, it is expected that 
a higher amount of ILP can be extracted
from serial integer programs.

Our research methodology is to create a new cycle accurate simulator
that will allow for various characterizations
of the several configurable machine features.  Configurable
elements include such things as the numbers of various units 
(issue stations, function units and types, et cetera) but also
the amount and configuration of various operand exchange bussing.
The simulator will have the ability to execute standard benchmark
programs using an existing instruction set architecture and
compiled using standard (including vendor supplied) compilers
and linkers.
%
%\bibliographystyle{latex8}
%\bibliography{ooo}
%
\end{document}
%
%
%
